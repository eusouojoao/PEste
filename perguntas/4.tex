\lstset{style=single1}
\clearpage
%%%%%%%%%%%%%%%%%%%%%%%%%%%%%%%%%%%%%%%%%%%%%%%%%%%%%%%%%%%%%%%%%%%%%%%%%%%%%%
%-1
\begin{lstlisting}[caption=Pergunta 4, frame=tlrb]{pergunta4}
pacman::p_load(pacman, rio, tidyverse, ggplot2, hrbrthemes) # Load req. packages
## Dados -----------------------------------------------------------------------
utentes <- import("~/Projects/git/personal/PEst/Code_env/Utentes.xlsx")
result <- utentes[, c(3,4)]
# Covariância e coeficiente de correlação linear
IMC=result$IMC
TAD=result$TAD
cov=cov(IMC, TAD); sprintf("Covariância: %.2f", cov)
corr=cor(IMC, TAD); sprintf("Coeficiente de correlação linear: %.2f", corr)
## Gráfico de dispersão --------------------------------------------------------
ggplot(result, aes(x=IMC, y=TAD)) +
  geom_point(size=2, shape=23, color = 'royalblue3') +
  ggtitle("Gráfico de dispersão entre o TAD e IMC")  +
  theme(plot.title=element_text(hjust=0.5, size=32, face="bold"), 
        axis.title=element_text(size=24),
        axis.text=element_text(size=16))+
  geom_smooth(method=lm, color = 'darkblue')
\end{lstlisting}
%-2
%\iffalse
\begin{figure}[H]
    \centering
    \includegraphics[width = 0.85\linewidth]{imag/4.png}
    %\caption{}
    \label{fig:pergunta4}
\end{figure}
%\fi
%-3
\paragraph{Comentário:}
Note-se que apesar da aproximação linear, há uma fraca correlação entre o IMC e o TAD, tal pode ser observado gráficamente através dos diversos pontos dos dados recolhidos que não seguem este padrão. É então possível concluir que apesar de se associar\footnotemark que um aumento do IMC leva a um aumento do TAD, os diversos pontos dispersos, provenientes dos dados recolhidos, suscitam uma relação que não é perfeitamente linear - sendo este o motivo pelo qual foi considerado que a correlação entre as duas variáveis é fraca/não ideal, por análise gráfica. A associação positiva entre os dados é amplamente estudada\footnotemark[1], e portanto, esperada.
\begin{table}[h!]
\centering
    \begin{tabular}{||c | c||} 
    \hline
    \textbf{Covariância} [$s_{xy}$] & \textbf{Coeficiente de correlação linear} [$r_{xy}$] \\ 
    \hline\hline
    24.94 & 0.57 \\ 
    \hline
    \end{tabular}
\end{table}

Assim, analizando a covariância e do coeficiente de correlação linear, é facilmente corroborado o visualizado, e expectável entre o binómio de conjuntos de dados. Visto que $s_{xy} > 0$, é natural a associação positiva observada (e documentada\footnotemark[1]); no entanto, verifica-se $r_{xy} \approx 0.57$, que ainda se encontra longe do valor ideal unitário.

\footnotetext{Dua S, Bhuker M, Sharma P, Dhall M, Kapoor S. Body mass index relates to blood pressure among adults. N Am J Med Sci. 2014 Feb;6(2):89-95. doi: 10.4103/1947-2714.127751. PMID: 24696830; PMCID: PMC3968571.}
%^^^^^^^^^^^^^^^^^^^^^^^^^^^^^^^^^^^^^^^^^^^^^^^^^^^^^^^^^^^^^^^^^^^^^^^^^^^^%