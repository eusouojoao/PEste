\lstset{style=single2}
\clearpage
%%%%%%%%%%%%%%%%%%%%%%%%%%%%%%%%%%%%%%%%%%%%%%%%%%%%%%%%%%%%%%%%%%%%%%%%%%%%%%
%-1
\begin{lstlisting}[caption=Pergunta 2, frame=tlrb]{pergunta2}
pacman::p_load(pacman, rio, tidyverse, ggplot2, hrbrthemes) # Load req. packages
## Dados -----------------------------------------------------------------------
esperanca_vida <- import("~/Projects/git/personal/PEst/Code_env/EsperancaVida.xlsx")
result <- esperanca_vida[c(48:65), c(1, 54, 55, 56, 88, 89, 90)]
result$...54 <- as.numeric(esperanca_vida$...54[48:65])
result$...55 <- as.numeric(esperanca_vida$...55[48:65])
result$...56 <- as.numeric(esperanca_vida$...56[48:65])
result$...88 <- as.numeric(esperanca_vida$...88[48:65])
result$...89 <- as.numeric(esperanca_vida$...89[48:65])
result$...90 <- as.numeric(esperanca_vida$...90[48:65])
# Rename das colunas
names(result)[names(result) == "...1"] <- "Anos"
names(result)[names(result) == "...56"] <- "Letónia Homens"
names(result)[names(result) == "...55"] <- "Itália Homens"
names(result)[names(result) == "...54"] <- "Irlanda Homens"
names(result)[names(result) == "...90"] <- "Letónia Mulheres"
names(result)[names(result) == "...89"] <- "Itália Mulheres"
names(result)[names(result) == "...88"] <- "Irlanda Mulheres"
gtemporal <- gather(result, "Países", "Esperança de vida à nascença",2:7)
## Gráfico temporal ------------------------------------------------------------
ggplot(gtemporal, aes(x=`Anos`, y=`Esperança de vida à nascença`, color=`Países`, 
                      group =`Países`)) +
  geom_point(size=2, shape=23) +
  geom_line() +
  labs(x="Anos", y="Idade") +
  scale_color_manual(values=c('blue', 'dodgerblue2', 'black', 'gray31', 'red', 
                              'tomato1')) + 
  labs(color='Países',fill='Países') + 
  ggtitle("Esperança de vida à nascença entre 2002 e 2019")  +
  theme_linedraw() +
  theme(plot.title = element_text(hjust = 0.5, size=32, face="bold"), 
        axis.title=element_text(size=24),
        axis.text=element_text(size=16),
        legend.title=element_text(size=24),
        legend.text=element_text(size=16)) 
\end{lstlisting}
%-2
%\iffalse
\begin{figure}[H]
    \centering
    \includegraphics[width = 0.85\linewidth]{imag/2.png}
    %\caption{}
    \label{fig:pergunta2}
\end{figure}
%\fi
%-3
\paragraph{Comentário:}
Independemente do género e do país, é natural concluir que a esperança média de vida tem aumentado ao longo dos anos, graças aos dados desencadeados. É verificável que a aparente \texit{trend} linear se mantém e que não se apresentam comportamentos cíclicos nem sazonais - o fenómeno é expectável, visto que o país de nascença afeta estridulamente a esperança de vida\footnotemark.

Por observação direta do gráfico temporal apresentado em seguida, destaca-se também que a esperança média de vida à nascença é superior para as mulheres em relação à dos homens em todos os casos. 

É também salientado que a Itália é o país com maior esperança média de vida do grupo analizado, seguido da Irlanda e, por final, pela Eslováquia (especialmente no caso dos homens, em que a diferença é aproximadamente 10 anos, em comparação com os outros dois países).

\footnotetext{Eric Neumayer and Thomas Plümper. Inequalities of income and inequalities of longevity: A cross-country study. American Journal of Public Health,
106(1):160–165, 2016. PMID: 26562120.}
%^^^^^^^^^^^^^^^^^^^^^^^^^^^^^^^^^^^^^^^^^^^^^^^^^^^^^^^^^^^^^^^^^^^^^^^^^^^^%