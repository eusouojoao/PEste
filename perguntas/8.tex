\lstset{style=single1}
\clearpage
%%%%%%%%%%%%%%%%%%%%%%%%%%%%%%%%%%%%%%%%%%%%%%%%%%%%%%%%%%%%%%%%%%%%%%%%%%%%%%
%-1
\begin{lstlisting}[caption=Pergunta 8, frame=tlrb]{pergunta8}
set.seed(172)
amostras=850
n=1060
lambda=0.67
options(digits=15)
## Calcular extremo da normal de lambda aproximada -----------------------------
gama=0.98
a=qnorm((1+gama)/2)
q=(1-gama)/2
a_tabela= 2.3263 #tabela utilizando o valor de q
mle <- rep(0,amostras)
avec <- rep(0,amostras)
for (i in 1:amostras){
  data <- rexp(n, lambda)
  mle[i] <- mean(data) # Valor esperado
  avec[i] <- (2*a)/(mle[i]*sqrt(n))
}
## Calcular amplitude ----------------------------------------------------------
amplitude = mean(avec); print(amplitude)
\end{lstlisting}
%-2
\iffalse
\begin{figure}[H]
    \centering
    \includegraphics[width = 0.85\linewidth]{imag/8.png}
    %\caption{}
    \label{fig:pergunta8}
\end{figure}
\fi
%-3

%^^^^^^^^^^^^^^^^^^^^^^^^^^^^^^^^^^^^^^^^^^^^^^^^^^^^^^^^^^^^^^^^^^^^^^^^^^^^%